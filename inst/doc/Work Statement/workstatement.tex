\documentclass[12pt,letterpaper]{article}

\usepackage{amsmath, amsthm, amssymb, amsfonts}
\usepackage{graphicx}
\usepackage{bm}
\usepackage{natbib}

\theoremstyle{definition}
\newtheorem{dfn}{Definition}

\begin{document}

% The numbers below controls the amount of space between the following sections
\def\shiftdowna{0.32in}  % Adjust for balance
\def\shiftdownb{0.22in}  % Adjust for balance

% Set up the boiler plate at the top of the page

\begin{center}
\textbf{{\large Project Work Statement}}\\


% SPONSOR
\vspace \shiftdowna
\underline {Sponsor}\\ 
\vspace{5pt}
\textbf{{\large Eastern States Interconnection Planning Council}}\\


% TITLE
\vspace \shiftdowna
\textbf{{\large The Value of Co-optimization in Electric Power Planning}}


% STUDENTS
\vspace{0.35in}
\vspace \shiftdownb
\underline {Participants} \\
\vspace{5pt}
\text{Jonathan Ho}, \texttt{jho19@jhu.edu}\\
\vspace{5pt}
\text{Jordan Mandel}, \texttt{jmande10@jhu.edu}\\
\vspace{5pt}
\text{Yue Wu}, \texttt{ywu67@jhu.edu}\\
\vspace{5pt}
\text{Mengshu Wang}, \texttt{mwang53@jhu.edu}\\
% SPONSORS
%\vspace \shiftdownb
%\underline {Potential Participants}\\
%\vspace{5pt}
%Youngser Park, \texttt{parky@jhu.edu} \\
%\vspace{3pt}
%\text{Mihn Tang}, \texttt{mtang10@jhu.edu} \\
%\vspace{3pt}
%\text{Glen Coppersmith}, \texttt{coppersmith@jhu.edu}

% DATE
\vspace \shiftdowna
Date: \today

\end{center}

\vfill  
%Fill page to force following note to bottom
\footnoterule
\noindent \small{Any apparent association of this work to  the real EISPC is
fictional, and the sole purpose of this work is a class exercise}

\newpage

\section{Background} 
The Eastern States Interconnection Planning Council (EISPC) is a collaborative organization for the Eastern AC Interconnection. Through it, state and Federal organizations work to improve coordination  amongst states and to develop better planning tools. EISPC facilitates interconnection wide planning decisions that will improve the robustness of the interconnection by making use of industry leading tools and expertise. 

\section{Problem Statement}
Using a co-optimization process that models the behavior of generators and transmission operators could reduce total system costs for both market participants.
There are two major issues that confound interconnection wide transmission planning efforts. The first is that organization of utilities transmission organizations in the United States. Traditionally in the utilities were vertically integrated publicly regulated monopolies. This allowed a utility to coordinate investments in generation and transmission. Under this type of organization there is little incentive for a utility to invest in transmission with a neighboring utility as their monopoly guaranteed that a utility could always afford to build the necessary generation capacity within their own network.

During the 90s deregulation of the power sector made it possible to break a utilities monopoly on generation. Under this model transmission would remain a monopoly operated by an independent system operator (ISO). ISOs operate a transmission network for a region and are nearly all non-profit. Generation is no longer a monopoly and utilities can either provide for themselves or buy from competitive independent generators. Under an ISO there is significantly more investment in transmission and interconnection as utilities will buy the cheapest available power in the system regardless of where it is produced. Making these interconnection decisions is complicated because the transmission planner, the ISOs, no longer coordinate with generation planers, utilities and independent producers.

An additional challenge to transmission planners is the expansion of renewable energy. Renewable generation has grown rapidly over the past decade and will continue to do so, with growth projections by the International Energy Agency in excess of 40\%, due to declining costs, government regulation, and economic incentives. \cite{IEA} In many cases the best quality renewable resources are located far from the areas of high demand, and would require significant investments in transmission to become viable. 

Currently the US electric network is made up of a mix of deregulated power pools and vertically integrated monopolies. Transmission planning is not preformed with appropriate considerations for the value of co-optimization. Under existing planning process transmission planning is preformed under the assumption that generation remains constant. Ignoring the generation planning process may result in the construction of lines that turn out to be unnecessary as well as limit the construction of inexpensive renewables. This will result in higher costs to all parties involved. This project will demonstrate that co-optimizing transmission investment decisions can reduce the total cost of building and operating an electric power system. 

\section{Approach}
Using the 13 bus McCalley network reduction of the United States transmission planning decisions will be modeled using the Hobbs and Van der Weije 2011 co-optimization model.\cite{Hobbs, PSERC} Model inputs including inputs sourced from the Energy Information Agency, independent system operators, and the National Renewable Energy Lab will be spatially aggregated using ArcGIS to function with the 13 bus network model.
Co-optimization will be compared with individual optimization, which will be modeled using a modification of the Hobbs and Van der Weije model. In this generation will be optimized assuming fixed transmission, then transmission will be separately optimized with fixed generation. A simulation of this electric power system and its operation costs will be preformed before a total cost of not co-optimizing is reported.
\section{Milestones}
We have the following major deadlines:
\begin{itemize}
    \item Work Statement due date, Oct 24, 2012,
    \item Progress Report due date, Nov 5, 2012,
    \item Final Presentation due date, Nov 6, 2012,
    \item Final Report due date, Nov 30, 2012.
    \item Final Deliverables due date, Dec 7, 2012
\end{itemize}

\newpage

\section{Deliverable}
\subsection{From Team to Sponsor} % (fold)
The following outputs are expected from this project:
\begin{itemize}
    \item Co-optimization model adapted to a 13 bus US network
    \item Appropriate parameters and constraints by region
    \item A simulation based model for independent transmission optimization
    \item Cost comparison between the two methods
	\item R packages used to analyze optimization results
	\item Documentation of methods and tools used
    \item Technical report and presentations summarizing the work 
\end{itemize}

\subsection{From Sponsor to Team} % (fold)

In order for our project to be of successful one, we will need:
\begin{itemize}
    \item Relevant data to the model (provided by sponsor as of 10/23)
	\begin{itemize}
		\item Physical network properties
		\item Generation and transmission costs
		\item Electric power demand
		\item Renewable resource availability
	\end{itemize}
    \item Computing resources
    \item Timely responses to inquiries
    \item Familiarity working with large data sets
\end{itemize}


\newpage
\bibliographystyle{plain}
\renewcommand\bibname{Selected Bibliography Including Cited Works}
\nocite{*}
\bibliography{Biblio}
% Add your bibliography to Contents
% Bibliography must come last.
\end{document}
